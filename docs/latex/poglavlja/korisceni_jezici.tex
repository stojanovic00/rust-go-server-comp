\section{Коришћени језици}

\subsection{\textit{Rust}}
 \textit{Rust} је модерни програмски језик који је осмишљен са циљем обезбеђивања сигурности, перформанси и практичности. Развијен од стране \textit{Mozilla research}-а \cite{mozillaResearch}, истиче се својом снажном подршком за паралелно и конкурентно програмирање, чиме омогућава програмерима да ефикасно искористе савремене вишејезгарне процесорске архитектуре. Овај језик је познат по својој јединственој карактеристици - власничком систему типова, који омогућава прецизно управљање меморијом без угрожавања безбедности. \textit{Rust} такође пружа богат скуп функционалности, укључујући алгебарске типове података, макро систем и једноставну синтаксу. Својом комбинацијом перформанси и безбедности, често се користи у различитим областима, од системског програмирања до веб развоја, нудећи програмерима снажан алат за изградњу поузданих и ефикасних софтверских решења.
 
\subsection{\textit{Golang}}
 \textit{Golang}, или \textit{Go}, је модерни програмски језик који је развијен у оквиру \textit{Google}-а. Основни принципи дизајна овог језика усмерени су ка једноставности, перформансама и ефикасности развоја софтвера. Истиче се брзим компајлирањем, чиме омогућава ефикасно испоручивање извршних фајлова, чак и за велике пројекте. Језик подржава конкурентно и паралелно програмирање као део свог језичког система, што га чини посебно прикладним за развој софтвера који захтева ефикасно управљање вишејезгарним процесорским архитектурама. \textit{Go} такође промовише чист и једноставан код, олакшавајући одржавање и разумевање софтверских пројеката. Често се користи у разним областима, укључујући серверски развој, мрежно програмирање и рад са контејнерима. Својим фокусом на продуктивност програмера и ефикасност извршавања кода, постао је популаран избор у индустрији софтверског инжењеринга.